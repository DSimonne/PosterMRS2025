% ============================================
% Third column, Results, key takeaways, references,
% acknowledgments
% ============================================

\separatorcolumn

\begin{column}{\colwidth}

    \begin{block}{X-ray irradiation strain relaxation effect}
        \begin{figure}
            \centering
            \includegraphics[width=\colwidth]{BeamEffect/Si_G/Si_G_Beam_effect_Poster.pdf}
            \caption{Strain relaxation of a Ni particle on \ce{SiO2}/\ce{Si}(001) from highly fluent X-ray beam visible by Bragg peak assymetry (top) and strain field (middle).
            The strain field energy (bottom) presents a quantitative decrease of the strain field inside the particle.
            }
        \end{figure}
    \end{block}

    \begin{block}{Corrosion in light water reactor environment}
        \begin{figure}
            \centering
            \includegraphics[width=\textwidth]{Corrosion/NiCo/MRSPoster.pdf}
            \caption{Strain buildup during corrosion of NiCo particle.}
        \end{figure}
    \end{block}

    \begin{block}{Defect signature in X-ray microscopy}
        \heading{Dark-field X-ray microscopy enables the visualisation of crystalline defects over extended ranges.}

        \begin{figure}
            \centering
            \includegraphics[width=\textwidth]{DFXM/Poster.pdf}
            \caption{DFXM image of Fe single crystal. Dislocations are visible over several hundreds of micrometers before hydrogen embrittlement.}
        \end{figure}
    \end{block}

    % \begin{block}{Scientific vision}
    %     \includegraphics[width=0.9\colwidth]{IntroductionMIT/Overall-research-vision.jpg}
    % \end{block}

    \begin{alertblock}{Key takeaways}
        \begin{itemize}
            \setlength\itemsep{1em}
            \item Synchrotron radiation allows \textit{operando} studies.
            \item Strain signature is accessible \textit{via} Bragg imaging techniques.
            \item Combining techniques allows nano as well as meso scale information.
        \end{itemize}

        \heading{Multi-scale approach, sensitive to material density, crystal structure, and defects, during \textit{operando} studies.}

    \end{alertblock}

    \begin{block}{References}
        \nocite{*}
        \footnotesize{\bibliographystyle{abbrv}\bibliography{references}}
    \end{block}

    \begin{block}{Acknowledgments}
        \centering
        \footnotesize{
            This work was funded by the Faculty Startup Fund support from MIT.
            The sample preparation work was carried out in part at MIT.nano, thanks to the help of Dr. Aubrey Penn and Juan Fererra.
            We acknowledge the ESRF and SOLEIL synchrotrons in France for providing beamtime utilizing beamline ID01 and SixS.
            We used the HPC at INL, supported by the Office of Nuclear Energy of the U.S. DOE and the NSUF under Contract No. DE-AC07-05ID14517.
        }

        \bigskip

        \includegraphics[height=2.5cm]{Logos/NSE/NSE_sub-brand_lockup_two-line_rgb_black.png}

    \end{block}

\end{column}

\separatorcolumn